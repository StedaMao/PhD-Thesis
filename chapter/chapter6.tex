%----------------------------------------------------------------------------------------
% CHAPTER SIX 
%----------------------------------------------------------------------------------------
\chapter{CONCLUSION AND OUTLOOK \label{ch:ch6}}
This thesis offers a comprehensive examination of nonlinear optical phenomena, focusing on the injection of photocarriers and the generation of high-order harmonics, all analyzed through the lens of microscopic electron dynamics. It begins by establishing the theoretical groundwork for understanding time-dependent quantum dynamics induced by light in solid systems, encompassing the intricate interaction between light and matter.

Chapter (\ref{ch:ch2}) introduces the tight-binding model, illustrating its application to typical hexagonal lattice nanostructures in 2D materials. The theoretical framework of the time-dependent Schrödinger equation (TDSE) and quantum master equation are introduced for simulating dynamical evolution. Through theoretical exploration, the thesis investigates the injection of dc-current and the emergence of population imbalances under two-color linearly polarized laser fields with frequencies ω and 2ω.

In Chapter (\ref{ch:ch3}), the focus shifts to the light-induced electron dynamics in a prototypical two-dimensional insulator, \gls{h-BN}, employing a simplified tight-binding approximation within a perturbative resonant regime. Surprisingly, even under deeply off-resonant conditions, the thesis reveals the induction of ballistic current by two-color linearly polarized light, showing a possibility for efficient electron population control without necessitating elliptical light polarization.

Chapter (\ref{ch:ch4}) delves into the mechanism of THz-induced high-order harmonic generation (HHG) and nonlinear electric transport in graphene. Utilizing the quantum master equation with the relaxation time approximation, the thesis provides a comprehensive understanding of these phenomena. Emphasis is placed on the pivotal role of nonequilibrium electron dynamics in conductivity reduction and the prevention of interband excitation.

In Chapter (\ref{ch:ch5}), the thesis explores strategies to enhance or suppress the efficiency of solid-state HHG, aiming to advance innovative HHG-based light sources and spectroscopies. Employing the quantum master equation, electron dynamics under both mid-infrared (MIR) and THz fields are analyzed, revealing the central role of coupling-induced coherence in enhancing MIR-induced HHG. This sheds light on the significance of field-induced coherence, transcending traditional population effects, and paving the way for future advancements in quantum dynamics and optoelectronic applications.

Based on our discussion, we show the tight-binding model offers several advantages in studying electronic and nonlinear optical response properties of materials:
\begin{itemize}
	\item Computational Efficiency: The tight-binding model provides a computationally efficient way to describe electronic structures compared to more elaborate methods like time-dependent density functional theory (TD-DFT). It reduces the computational CPU time while still capturing essential electronic properties.

	\item Parameterization Flexibility: The model allows for flexibility in parameterization, enabling researchers to tailor it to specific material systems or phenomena of interest. This flexibility makes it suitable for studying a wide range of materials, including complex structures and heterostructures.

	\item Insight into Electronic Structure: By explicitly considering the atomic interactions within a material, the tight-binding model provides valuable insights into its microscpic electronic structure. This helps in understanding how electrons behave within the material and how their properties influence optical responses.

	\item Interpretability: The simplicity of the tight-binding model often leads to more interpretable results. This makes it easier to identify the underlying physical mechanisms governing electronic and optical phenomena in materials, facilitating a deeper understanding of their behavior.
\end{itemize}

Overall, our work advances the theoretical framework for studying nonlinear optics in 2D materials and provides guidance for future experimental studies and device design by solving quantum dynamics based on tight-binding model, which holds promising prospects for advancing nonlinear optical technology and ultrafast techniques, offering practical applications and avenues for further exploration:
\begin{itemize}
	\item Exploring New Materials: Building upon the comprehensive examination of nonlinear optical phenomena in hexagonal boron nitride and graphene, future research could extend our tight-binding model to other materials such as transition metal dichalcogenides (TMDs) and black phosphorus and also 3D bulks. Also we can extend our 2-by-2 Hamiltonian to more bands to include more orbitals' impacts.

	\item Integrating External Stimuli: Investigating how other stimuli, like cavity or magnetic fields, influence nonlinear optical responses in 2D materials, with a focus on experimental validation to complement theoretical predictions.

	\item Experimental Validation: Experimental validation of theoretical predictions is crucial for advancing our understanding of nonlinear optical phenomena in 2D materials. By employing advanced spectroscopic techniques and ultrafast laser spectroscopy, we can directly probe these phenomena under controlled conditions to cooperate with experimentalists.

	\item Multi-Physics Modeling: Integrating multi-physics effects like electron-phonon interactions into theoretical models will provide a more comprehensive understanding of nonlinear optical phenomena. This approach will be essential for accurately predicting material behavior under realistic operating conditions.

	\item Quantum Dynamics Exploration: Exploring nonlinear quantum dynamics in highly excited states of 2D materials presents an opportunity to uncover new quantum phenomena. This could lead to the development of innovative quantum technologies based on nonlinear optics in 2D materials.

\end{itemize}
Through these interconnected research directions, our future work can advance understanding and utilization of nonlinear optical phenomena not only in 2D materials, but also offer unprecedented opportunities for theoretical prediction, innovation, and scientific discovery across various material systems, paving the way for transformative advancements in optoelectronics and beyond.
