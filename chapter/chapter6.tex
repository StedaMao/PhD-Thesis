%----------------------------------------------------------------------------------------
% CHAPTER SIX 
%----------------------------------------------------------------------------------------
\chapter{CONCLUSION AND OUTLOOK \label{ch:ch6}}
This thesis offers a comprehensive examination of nonlinear optical phenomena, focusing on the injection of photocarriers and the generation of high-order harmonics, in 2d systems all analyzed through the lens of microscopic electron dynamics. It begins by establishing the theoretical groundwork for understanding time-dependent quantum dynamics induced by light in solid systems, encompassing the light and matter interactions. We emphasize the theoretical nonequilibrium framework by presented tight-binding approach in this thesis for analyzing nonlinear optical phenomena in materials, particularly in microscopic details and conducting dissipative non-equilibrium analyses. By synthesizing these discoveries, the thesis contributes to the advancement of our comprehension of nonlinear optical phenomena in 2D materials, highlighting the importance of nonequilibrium quantum dynamics in modeling complex behaviors:

Chapter (\ref{ch:ch2}) introduces the tight-binding model, illustrating its application to typical hexagonal lattice nanostructures in 2D materials. The theoretical framework of the time-dependent Schrödinger equation (TDSE) and quantum master equation are introduced for simulating dynamical evolution. Through theoretical exploration, the thesis investigates the injection of dc-current and the emergence of population imbalances under two-color linearly polarized laser fields with frequencies ω and 2ω.

In Chapter (\ref{ch:ch3}), the focus shifts to the light-induced electron dynamics in a prototypical two-dimensional insulator, \gls{h-BN}, employing a simplified tight-binding approximation within a perturbative resonant regime. Surprisingly, even under deeply off-resonant conditions, the thesis reveals the induction of ballistic current by two-color linearly polarized light, showing a possibility for efficient electron population control without necessitating elliptical light polarization.

Chapter (\ref{ch:ch4}) delves into the mechanism of THz-induced high-order harmonic generation (HHG) and nonlinear electric transport in graphene. Utilizing the quantum master equation with the relaxation time approximation, the thesis provides a comprehensive understanding of these phenomena. Emphasis is placed on the pivotal role of nonequilibrium electron dynamics in conductivity reduction and the prevention of interband excitation.

In Chapter (\ref{ch:ch5}), the thesis explores strategies to enhance or suppress the efficiency of solid-state HHG, aiming to advance innovative HHG-based light sources and spectroscopies. Employing the quantum master equation, electron dynamics under both mid-infrared (MIR) and THz fields are analyzed, revealing the central role of coupling-induced coherence in enhancing MIR-induced HHG. This sheds light on the significance of field-induced coherence, transcending traditional population effects, and paving the way for future advancements in quantum dynamics and optoelectronic applications.

\color{red}
The results presented in those chapters and in the published articles this PhD work has produced (see Publication Lists).

Based on our discussion, we show the tight-binding model a valid approach with computational efficiency in studying electronic and nonlinear optical response properties of materials. While it provides a simple and intuitive description of electronic band structures, its accuracy may be limited due to its semi-empirical nature and reliance on fitting parameters. However, ab initio Density Functional Theory (DFT) serves as a foundation for improving the tight-binding model. By approximating first principles models derived from DFT, the tight-binding model benefits from the accuracy of DFT while achieving computational efficiency through reduced Hilbert space size. This approximation acknowledges the underlying first principles modeling and maintains close ties to exact results, enhancing its applicability in predicting electronic properties of materials.

The tight-binding model introduced in this thesis only considering the basic nearest hopping and only two bands (one valence band and one conduction band), which can be expanded to more onsite interactions like next-nearest hopping and even more bands. Tight-binding model allows for the inclusion of various physical effects such as electron-electron interactions, electron-phonon interactions, and external fields. This flexibility enables researchers to tailor the model to specific material systems or phenomena of interest, making it suitable for studying a wide range of materials, including complex structures and heterostructures.

The tight-binding model also provides valuable insights into the microscopic electronic structure of materials, facilitating a deeper understanding of their behavior. By explicitly considering atomic interactions within a material, the model offers a more interpretable description of electronic properties, making it easier to identify the underlying physical mechanisms governing electronic and optical phenomena. This interpretability enhances the model's utility in guiding experimental studies and device design, enabling researchers to make informed decisions based on fundamental principles.

Tight-binding model is an approximation to the first principles models that can be drerived from ab-initio DFT calculations and used to improved computational effeciency through Wannierization, which represents a systematic approach to constructing tight-binding models from first principles calculations. Wannierization facilitates the transformation of electronic wavefunctions from a basis of Bloch states to localized Wannier functions. Wannier functions provide a clearer physical interpretation of the electronic structure compared to Bloch states. Each Wannier function corresponds to an electron localized around a particular atomic site within the crystal lattice. This localization allows for a more intuitive understanding of electronic properties, such as onsite energy, bonding, hopping, and interactions, in terms of localized atomic-like orbitals. Wannierization can help educe the size of the Hilbert space, as the model focuses on the interactions between a limited number of localized orbitals rather than considering the entire Brillouin zone; provide a systematic method for constructing the tight-binding Hamiltonian; allow for the localization of these interaction terms around specific lattice sites, making the tight-binding model more physically intuitive and interpretable; moreover, enable the incorporation of additional physical effects, such as electron-electron interactions, spin-orbit coupling, and lattice distortions, into the tight-binding model.

\color{black}
Overall, our work advances the theoretical framework for studying nonlinear optics in 2D materials and provides guidance for future experimental studies and device design by solving quantum dynamics based on tight-binding model, which holds promising prospects for advancing nonlinear optical technology and ultrafast techniques, offering practical applications and avenues for further exploration:
\begin{itemize}
	\item \textcolor{red}{While we introduce the topological phase and Berry curvature in Chapter.~(\ref{ch:ch2}) for 2d topological insulators, we also expande Houston states into dynamicl phase and geometric phase under adibiatic approximation, we didn't go further for the time-dependent perturbation analysis by considering the contribution from geometric phase. The light-induced topological phase transition and the Berry curvature effects can be further studied in the future both analytically and numerically.}

	\item \textcolor{red}{The relaxation approximation we introduced in Chapter.~(\ref{ch:ch2}) is a simple and effective method to consider the dissipative effects in the quantum master equation, we can further explore the non-Markovian effects in the relaxation approximation to consider the memory effects in the dissipative dynamics. Also, the relaxation time $T_1$ and $T_2$ are wrote down artificially in the model, we can consider more realistic relaxation time from the experiments or from first principles calculations.}

	\item Exploring new materials: Building upon the comprehensive examination of nonlinear optical phenomena in hexagonal boron nitride and graphene, future research could extend our tight-binding model to bilayer, trilayer systems with stackling discussion; in addition, to other materials such as transition metal dichalcogenides (TMDs) and black phosphorus and also 3D bulks. \textcolor{red}{More validation models can be constructed by Wannierization from DFT as we introduced, we can study the nonlinear optical response on Moiré systems by the construction of continuum models for twisted systems which can be more efficient in computation than \gls{TDDFT}.}

	\item Integrating external stimuli: Investigating how other stimuli, like cavity or magnetic fields, influence nonlinear optical responses in 2D materials, with a focus on experimental validation to complement theoretical predictions. \textcolor{red}{Quantum Electrodynamics Density Functional Theory (QEDFT) provides a suitable theoretical framework for investigating such processes. QEDFT combines the principles of quantum mechanics and electrodynamics, allowing for the accurate description of the interactions between electrons and electromagnetic fields. The tight-binding model can be integrated with QEDFT to provide a more comprehensive understanding of the system's response to external stimuli.}

	\item Experimental validation: Experimental validation of theoretical predictions is crucial for advancing our understanding of nonlinear optical phenomena in 2D materials. By employing advanced spectroscopic techniques and ultrafast laser spectroscopy, we can directly probe these phenomena under controlled conditions \textcolor{red}{for the validate theoretical predictions in the experiments such as: Transient Absorption Spectroscopy, Second Harmonic Generation (SHG) Spectroscopy, Pump-Probe Spectroscopy like Angle-resolved photoemission spectroscopy (ARPES), and Attosecond Ultrafast Techniques.}

	\item \textcolor{red}{Integrate with Wannierization interface: As we introduced before, with the interface of Wannierization, the tight-binding model can be further improved by incorporating more realistic electronic structures and interactions. This enhancement will enable the model to capture a wider range of physical effects and provide more accurate predictions of material properties by connecting mature ab initio packages like VASP, QE with Wannier90, even allow us to do topological anaylsis by WannierTools. This interface might also help adding dissipation effect for quantum dynamic computation in the open system is also expexted to be conducted in \gls{TDDFT} in OCTOPUS from a reduced Hilbert space.}

\end{itemize}
Through these interconnected research directions, our future work can advance understanding and utilization of nonlinear optical phenomena not only in 2D materials, but also offer unprecedented opportunities for theoretical prediction, innovation, and scientific discovery across various material systems, paving the way for transformative advancements in optoelectronics and beyond, to material science and ultrafast techniques developments.
