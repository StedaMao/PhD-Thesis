%----------------------------------------------------------------------------------------
% CHAPTER SIX 
%----------------------------------------------------------------------------------------
\chapter{CONCLUSION AND OUTLOOK \label{ch:ch6}}


This thesis explored nonlinear optical phenomena in two-dimensional (2D) materials in typical hexagonal lattices, using a tight-binding model combined with the time-dependent Schrödinger equation and quantum master equation. Through theoretical analysis and simulations, we gained insights into the microscpic mechanism behind photovoltaic effects in \gls{h-BN} and HHG in graphene. Our study contributes to understanding how 2D materials behave under different and composite light conditions. By using the tight-binding model, we accurately captured their electronic structures and how they change when exposed to light. Solving the time-dependent Schrödinger equation helped us simulate how electrons move within these materials under light, revealing their nonlinear optical responses. Additionally, the quantum master equation helped us understand the processes of relaxation and decoherence, which are important for interpreting the observed optical phenomena under strong field conditions like \gls{HHG}.

We show the tight-binding model offers several advantages in studying electronic and nonlinear optical response properties of materials:
\begin{itemize}
	\item Computational Efficiency: The tight-binding model provides a computationally efficient way to describe electronic structures compared to more elaborate methods like time-dependent density functional theory (TD-DFT). It reduces the computational CPU time while still capturing essential electronic properties.

	\item Parameterization Flexibility: The model allows for flexibility in parameterization, enabling researchers to tailor it to specific material systems or phenomena of interest. This flexibility makes it suitable for studying a wide range of materials, including complex structures and heterostructures.

	\item Insight into Electronic Structure: By explicitly considering the atomic interactions within a material, the tight-binding model provides valuable insights into its microscpic electronic structure. This helps in understanding how electrons behave within the material and how their properties influence optical responses.

	\item Interpretability: The simplicity of the tight-binding model often leads to more interpretable results. This makes it easier to identify the underlying physical mechanisms governing electronic and optical phenomena in materials, facilitating a deeper understanding of their behavior.
\end{itemize}

For further connecting the tight-binding model to time-dependent density functional theory (TD-DFT) simulations and experiments involves several steps:
\begin{itemize}
	\item    Parameter Calibration: The parameters of the tight-binding model can be calibrated using experimental data or results from more accurate theoretical methods like DFT. This ensures that the model accurately captures the material's electronic structure and optical properties.

	\item Correlation with TDDFT: TDDFT simulations provide detailed information about the material's response to external perturbations, such as light. By comparing the results of tight-binding simulations with TDDFT calculations, researchers can validate the accuracy of the tight-binding model and identify any discrepancies that need to be addressed.

	\item Prediction and Analysis: Once validated, the tight-binding model can be used to make predictions about the material's behavior under different conditions or in scenarios that may be challenging to study directly with TDDFT. These predictions can then be analyzed and compared with experimental observations to further validate the model and gain insights into the material's properties.

	\item Experimental Verification: Experimentalists can use the insights gained from tight-binding simulations to design experiments aimed at probing specific electronic or optical properties predicted by the model. The experimental results can then be compared with the theoretical predictions, providing further validation and refinement of the model.
\end{itemize}
Overall, our work advances the theoretical framework for studying nonlinear optics in 2D materials and provides guidance for future experimental studies and device design by solving quantum dynamics based on tight-binding model, which holds promising prospects for advancing nonlinear optical technology and ultrafast techniques, offering practical applications and avenues for further exploration.
