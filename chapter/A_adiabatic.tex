\chapter{ADIABATIC BASIS REPRESENTATION \label{ch:A_ADIABATIC}}
\label{ch:Adiabatic}
To study the mechanism behinde the nonlinear injection of photocarriers in solids from an analytical perspective, we initially present the equation of motion within the adiabatic basis framework. Within this framework, we can effectively distinguish between intraband and interband transitions, the dynamic phase factor, and the geometric phase factor. To establish this framework, we start by considering the following one-body Schrödinger equation for a $\vecb k$-point same as Eq.~(\ref{eqn:TDSE}):

\begin{equation}
	i \frac{d}{dt} | \psi_{\mathbf k}(t) \rangle = H \left [\mathbf k + \mathbf A(t) \right ] | \psi_{\mathbf k}(t) \rangle,
	\label{eqn:tdse00}
\end{equation}

Here, $\vecb A(t)$ represents an external vector potential, which correlates with the external electric field as $\vecb E(t)=-d\vecb A(t)/dt$. Throughout this discussion, we presume that the vector potential remains zero for negative times, i.e., $\vecb A(t\leq 0)=0$.

In the adiabatic basis representation, we define the instantaneous eigenstates of the Hamiltonian:

\begin{align}
	H\left [\vecb k + \vecb A(t) \right ] |u_{b,\vecb k+\vecb A(t)}\rangle = \epsilon_{b,\vecb k + \vecb A(t)}|u_{b,\vecb k+\vecb A(t)}\rangle,
	\label{eq:static-se}
\end{align}

Here, $b$ denotes the band index. For simplicity, we consider a two-band system comprising the valence band ($b=v$) and the conduction band ($b=c$). Nonetheless, this formulation can be straightforwardly extended to encompass general systems.

We can do the following expansion of wavefunction based on the instantaneous eigenstates defined by Eq.~(\ref{eq:static-se}), which following the same expasion of the single-particle orbital in terms of Houston states \cite{PhysRev.57.184, PhysRevB.33.5494, hirori2024high}:
\begin{align}
	|\psi_{\mathbf k}(t)\rangle = c_{v,\mathbf k}(t) e^{-i\int^t_0dt' \epsilon_{v,\mathbf k+ \mathbf A(t')}} e^{i\phi^g_{v,\mathbf k}(t)} |u_{v,\mathbf k+\mathbf A(t)}\rangle
	+c_{c,\mathbf k}(t) e^{-i\int^t_0dt' \epsilon_{c,\mathbf k+ \mathbf A(t')}} e^{i\phi^g_{c,\mathbf k}(t)} |u_{c,\mathbf k+\mathbf A(t)}\rangle,
	\label{eq:ansatz}
\end{align}
The expansion involves coefficients $c_{b,\vecb k}(t)$ for each band. Explicitly, we include the dynamical phase factor $e^{-i\int^t_0dt' \epsilon_{v,\vecb k+ \vecb A(t')}}$ and introduce an additional phase factor $e^{i\phi^g_{b,\vecb k}(t)}$ as explained in Section.~(\ref{sec:deriveperturbation}). The latter will be defined subsequently as the geometric phase factor following the analytical steps in Ref.~(\cite{hirori2024high}).

Inserting expansion Eq.~(\ref{eq:ansatz}) into Eq.~(\ref{eqn:tdse00}), we have:
\begin{align}
	\left [i\frac{d}{dt}-H\left [\mathbf k + \mathbf A(t) \right ] \right ]|\psi_{\mathbf k}(t) \rangle & =
	i \dot c_{v,\mathbf k}(t) e^{-i\int^t_0dt' \epsilon_{v,\mathbf k+ \mathbf A(t')}} e^{i\phi^g_{v,\mathbf k}(t)} |u_{v,\mathbf k+\mathbf A(t)}\rangle \nonumber                                                                                                                                                          \\
	                                                                                                    & + i\dot c_{c,\mathbf k}(t) e^{-i\int^t_0dt' \epsilon_{c,\mathbf k+ \mathbf A(t')}} e^{i\phi^g_{c,\mathbf k}(t)} |u_{c,\mathbf k+\mathbf A(t)}\rangle \nonumber                                                   \\
	                                                                                                    & - \dot \phi^g_{v,\mathbf k}(t) c_{v,\mathbf k}(t) e^{-i\int^t_0dt' \epsilon_{v,\mathbf k+ \mathbf A(t')}} e^{i\phi^g_{v,\mathbf k}(t)} |u_{v,\mathbf k+\mathbf A(t)}\rangle \nonumber                            \\
	                                                                                                    & - \dot \phi^g_{c,\mathbf k}(t) c_{c,\mathbf k}(t) e^{-i\int^t_0dt' \epsilon_{c,\mathbf k+ \mathbf A(t')}} e^{i\phi^g_{c,\mathbf k}(t)} |u_{c,\mathbf k+\mathbf A(t)}\rangle \nonumber                            \\
	                                                                                                    & - i c_{v,\mathbf k}(t) e^{-i\int^t_0dt' \epsilon_{v,\mathbf k+ \mathbf A(t')}} e^{i\phi^g_{v,\mathbf k}(t)} \mathbf E(t)\cdot \frac{\partial |u_{v,\mathbf k+\mathbf A(t)}\rangle}{\partial \mathbf k} \nonumber \\
	                                                                                                    & - i c_{c,\mathbf k}(t) e^{-i\int^t_0dt' \epsilon_{c,\mathbf k+ \mathbf A(t')}} e^{i\phi^g_{c,\mathbf k}(t)} \mathbf E(t)\cdot \frac{\partial |u_{c,\mathbf k+\mathbf A(t)}\rangle}{\partial \mathbf k} = 0.
	\label{eq:tdse01}
\end{align}

By multiplying $e^{+i\int^t_0dt' \epsilon_{v,\mathbf k+ \mathbf A(t')}} e^{-i\phi^g_{v,\mathbf k}(t)} \langle u_{v,\mathbf k+\mathbf A(t)}|$ to Eq.~(\ref{eq:tdse01}), we have:
\begin{align}
	 & i \dot c_{v,\mathbf k}(t) - \dot \phi^g_{v,\mathbf k}(t) c_{v,\mathbf k}(t) - i c_{v,\mathbf k}(t) \mathbf E(t)\cdot \left \langle u_{v,\mathbf k+\mathbf A(t)}\Big |\frac{\partial u_{v,\mathbf k+\mathbf A(t)}}{\partial \mathbf k} \right \rangle \nonumber \\
	 & - i c_{c,\mathbf k}(t) e^{-i\int^t_0dt' \epsilon_{c,\mathbf k+ \mathbf A(t')}-\epsilon_{v,\mathbf k+ \mathbf A(t')}} e^{i\left (\phi^g_{c,\mathbf k}(t)-\phi^g_{v,\mathbf k}(t)\right )}
	\mathbf E(t)\cdot \left \langle u_{v,\mathbf k+\mathbf A(t)}\Big |\frac{\partial u_{c,\mathbf k+\mathbf
	A(t)}}{\partial \mathbf k} \right \rangle                                                                                                                                                                                                                         \\& = 0.
	\label{eq:eom-valence}
\end{align}

Similarly, by multiplying $e^{+i\int^t_0dt' \epsilon_{c,\mathbf k+ \mathbf A(t')}} e^{-i\phi^g_{c,\mathbf k}(t)} \langle u_{c,\mathbf k+\mathbf A(t)}|$ to Eq.~(\ref{eq:tdse01}), we have:
\begin{align}
	 & i \dot c_{c,\mathbf k}(t) - \dot \phi^g_{c,\mathbf k}(t) c_{c,\mathbf k}(t) - i c_{c,\mathbf k}(t) \mathbf E(t)\cdot \left \langle u_{c,\mathbf k+\mathbf A(t)}\Big |\frac{\partial u_{c,\mathbf k+\mathbf A(t)}}{\partial \mathbf k} \right \rangle \nonumber \\
	 & - i c_{v,\mathbf k}(t) e^{-i\int^t_0dt' \epsilon_{v,\mathbf k+ \mathbf A(t')}-\epsilon_{c,\mathbf k+ \mathbf A(t')}} e^{i\left (\phi^g_{v,\mathbf k}(t)-\phi^g_{c,\mathbf k}(t)\right )}
	\mathbf E(t)\cdot \left \langle u_{c,\mathbf k+\mathbf A(t)}\Big |\frac{\partial u_{v,\mathbf k+\mathbf A(t)}}{\partial \mathbf k} \right \rangle = 0.
	\label{eq:eom-conduction}
\end{align}

We have the following matrix form for time-dependnt expansion coefficient vector combining Eq.~(\ref{eq:eom-valence}) and Eq.~(\ref{eq:eom-conduction}):
\begin{align}
	 & i\frac{d}{dt} \mathbf c_{\mathbf k}(t) =
	\left(
	\begin{array}{cc}
			\dot \phi^g_{v,\mathbf k}(t) & 0                            \\
			0                            & \dot \phi^g_{c,\mathbf k}(t)
		\end{array}
	\right) \mathbf c_{\mathbf k}(t)
	+i\mathbf E(t)\cdot \left(
	\begin{array}{cc}
			M_{11} & M_{12} \\
			M_{21} & M_{22}
		\end{array}
	\right) \mathbf c_{\mathbf k}(t),
\end{align}

\begin{align}
	 & M_{11} = \left \langle u_{v,\mathbf k+\mathbf A(t)}\Big |\frac{\partial u_{v,\mathbf k+\mathbf
	A(t)}}{\partial \mathbf k} \right \rangle                                                                                                   \\
	 & M_{12} = e^{-i\int^t_0dt' \Delta \epsilon_{cv,\mathbf k+ \mathbf A(t')}+i \Delta \phi^g_{cv,\mathbf k}(t)}
	\left \langle u_{v,\mathbf k+\mathbf A(t)}\Big |\frac{\partial u_{c,\mathbf k+\mathbf A(t)}}{\partial
	\mathbf k} \right \rangle                                                                                                                   \\
	 & M_{21} = e^{-i\int^t_0dt' \Delta \epsilon_{vc,\mathbf k+ \mathbf A(t')}+i \Delta \phi^g_{vc,\mathbf k}(t)}
	\left \langle u_{c,\mathbf k+\mathbf A(t)}\Big |\frac{\partial u_{v,\mathbf k+\mathbf A(t)}}{\partial \mathbf k} \right \rangle             \\
	 & M_{22} = \left \langle u_{v,\mathbf k+\mathbf A(t)}\Big |\frac{\partial u_{v,\mathbf k+\mathbf A(t)}}{\partial \mathbf k} \right \rangle
\end{align}

The coefficient vector was defined as:
\begin{align}
	\mathbf c_{\mathbf k}(t) = \left(
	\begin{array}{cc}
			c_{v,\mathbf k}(t) \\
			c_{c,\mathbf k}(t)
		\end{array}
	\right).
\end{align}
where $\Delta\epsilon_{bb',\mathbf k+ \mathbf A(t)}$ is defined by the difference of the single particle energies as $\epsilon_{b,\mathbf k+ \mathbf A(t)}-\epsilon_{b',\mathbf k+ \mathbf A(t)}$, and $\Delta \phi^g_{bb',\mathbf k}(t)$ is defined by the difference of the geometric phases as $\phi^g_{b,\mathbf k}(t)-\phi^g_{b',\mathbf k}(t)$.

Here, we define the geometric phases as
\begin{align}
	\phi^g_{b,\mathbf k}(t) & = -i\int^t_0 dt' \mathbf E(t')\cdot
	\left \langle u_{b,\mathbf k+\mathbf A(t')}\Big |\frac{\partial u_{b,\mathbf k+\mathbf A(t')}}{\partial
	\mathbf k} \right \rangle \nonumber                                                                                                                                                                           \\
	                        & =i\int^t_0 dt' \frac{dA(t')}{dt'}\cdot
	\left \langle u_{b,\mathbf k+\mathbf A(t')}\Big |\frac{\partial u_{b,\mathbf k+\mathbf A(t')}}{\partial \mathbf k} \right \rangle \nonumber                                                                   \\
	                        & = i \oint^{\mathbf A(t)}_{\mathbf A(0)} d\mathbf A \cdot \left \langle u_{b,\mathbf k+\mathbf A}\Big |\frac{\partial u_{b,\mathbf k+\mathbf A}}{\partial \mathbf k} \right \rangle.
	\label{eq;geometric-phase}
\end{align}
As observed in the final expression of Eq.~(\ref{eq;geometric-phase}), the phase $\phi^g_{b,\vecb k}$ only relies on the geometry of the integral path. For simplicity, we presume that contributions from the geometric phases, denoted as $\Delta \phi^g_{cv,\vecb k}(t)$, are negligible in the perturbation analysis in Section.~(\ref{sec:deriveperturbation}). This assumption is valid for the two-band tight-binding model for hexagonal lattices under our discussion in the thesis, which have particle-hole symmetry.

With the expression of the geometric phases in Eq.~(\ref{eq;geometric-phase}), it becomes simplified to rewrite the equation of motion for the coefficient vector as:
\begin{align}
	 & i\frac{d}{dt} \mathbf c_{\mathbf k}(t) = \mathcal{H}(t) \mathbf c_{\mathbf k}(t)
	.
	\label{eq:tdse-ad-basis_A}
\end{align}
\begin{align}
	 & \mathcal{H}(t)=
	i\mathbf E(t)\cdot \left(
	\begin{array}{cc}
			0      & M_{12} \\
			M_{21} & 0
		\end{array}
	\right)
\end{align}
This equation is essentially the time-dependent Schrödinger equation within the adiabatic basis, closely linked to the Houston basis expansion \cite{PhysRev.57.184,PhysRevB.33.5494}.
