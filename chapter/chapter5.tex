\chapter{ENHANCEMENT OF MIR-INDUCED HHG BY COHERENT COUPLING WITH THZ FIELD \label{ch:ch5}}

Following a detailed exploration of the phenomenon of DC-current injection and the generation of
population imbalance through the application of two-color linearly polarized laser fields in
Chapter~\ref{ch:ch2}, it is also imperative to enhance the efficiency of solid-state High Harmonic Generation (HHG) for the development of innovative HHG-based light sources and spectroscopies. Recent investigations have indicated the potential enhancement of HHG from graphene using two-color laser fields, leveraging various mechanisms~\cite{PhysRevB.100.035434,Mrudul:21,PhysRevB.105.195405}.

In the work by Ref.\cite{PhysRevB.100.035434}, the concept of two-color HHG is proposed, involving the interplay of electron-hole pair creation induced by high-frequency pump light and the subsequent acceleration of these created pairs by low-frequency light. Mrudul \textit{et al.} delved into HHG from graphene under bicircular fields, showcasing the ability to control valley polarization\cite{Mrudul:21}. Additionally, Avetissian~\textit{et al.} explored HHG from graphene under linearly polarized light and its second harmonics. They demonstrated that, when the two-color fields are perpendicularly polarized, stronger harmonics can be emitted compared to parallel polarization~\cite{PhysRevB.105.195405}.
Recently, High Harmonic Generation (HHG) from graphene has garnered experimental attention in the
mid-infrared (MIR)\cite{doi:10.1126/science.aam8861,cha2022gate} and terahertz
(THz)\cite{Hafez2018,doi:10.1126/sciadv.abf9809} regimes, revealing distinctive ellipticity
dependence and remarkable efficiency. Similar to our previous exploration of HHG from graphene in
the THz regimes in Chapter~\ref{ch:ch4}, based on a quantum master equation, this theoretical methodology has been adeptly applied in the MIR region~\cite{PhysRevB.103.L041408} for the elucidation of experimental results~\cite{doi:10.1126/science.aam8861,cha2022gate}.

Furthermore, in the MIR regime, the coupling between field-induced intraband and interband transitions unfolds crucial channels for HHG, leading to enhanced HHG with finite ellipticity~\cite{PhysRevB.103.L041408}. Real-time electron dynamics simulations in the THz regime have underscored the significance of considering the nonequilibrium steady-state, resulting from the delicate balance between field-driving and relaxation. This approach transcends the confines of the equilibrium thermodynamic framework and provides a more comprehensive understanding of HHG from graphene~\cite{PhysRevB.106.024313}.

In this chapter, we delve into the prospect of leveraging a terahertz (THz) field to augment mid-infrared (MIR)-induced High Harmonic Generation (HHG) in graphene, drawing insights from our cumulative knowledge. Initially, we employ a quantum master equation to scrutinize electron dynamics under both MIR and THz fields, subsequently assessing the emitted harmonic spectra. The outcomes from fully dynamical calculations are juxtaposed against a thermodynamic model that incorporates the equilibrium Fermi--Dirac distribution. Additionally, a nonequilibrium population model is considered, accounting for a population distribution in a nonequilibrium steady-state. Through our analysis, we unveil the pivotal role played by coupling induced coherence via THz and MIR fields in enhancing MIR-induced HHG. This elucidation underscores the significance of field-induced coherence, extending beyond mere population effects.
%=======================================================================================================================================================================
\section{MIR-induced HHG in Graphene under THz Fields}
%=======================================================================================================================================================================

In our analysis of High Harmonic Generation (HHG) induced by a Mid-Infrared (MIR) laser pulse in the presence of Terahertz (THz) fields, we adopt a practical form for the MIR pulse, expressed as follows:
\begin{align}
  \vecb A_{MIR}(t) = -\frac{E_{MIR}}{\omega_{MIR}} \vecb{e}_{MIR} 
\sin(\omega_{MIR} t) \cos^4 \left (\frac{\pi}{T_{MIR}} t \right)
  \label{eqn:laser_pulse}
\end{align}
This pulse is defined in the domain $-T_{MIR}/2<t<T_{MIR}/2$ and is zero outside this range.
Here, $E_{MIR}$ represents the peak strength of the MIR field, $\omega_{MIR}$ is the mean
frequency, $\vecb e_{MIR}$ is a unit vector indicating the polarization direction of light, and
$T_{MIR}$ is the pulse duration. Specifically, we set the pulse duration $T_{MIR}$ to 0.4 ps and
the mean frequency $\omega_{MIR}$ to 0.35424 eV/$\hbar$ for this study, while other parameters are
varied in our computation.

First, we investigate HHG in graphene only with the MIR fields. For practical analysis, the
direction of the angle $0^\circ$ is fixed to the $\Gamma$--$M$ axis (the $x-$axis in our setup), and the peak field strength of the MIR field $E_{MIR}$ is fixed at 6.5~MV/cm. The emitted harmonics are investigated by manipulating the polarization direction of the MIR field, $\vecb e_{MIR}$.

To analyze the HHG efficiency, we compute the signal intensity of the emitted harmonics from
Eq~(\ref{eqn:spectrum})at each
order by integrating the power spectrum within a finite energy range as  the integrated intensity
of the emitted $n$th harmonic $I^{n \textrm{th}}_{\mathrm{total}}$:
\begin{align}
I^{n \textrm{th}}_{\mathrm{total}} = \int_{\left (n-\frac{1}{2} \right )\omega_{MIR}}^{\left (n+\frac{1}{2} \right )\omega_{MIR}} d \omega I_{\textrm{HHG}} (\omega).
\label{eqn:integrate_intensity}
\end{align}

In our evaluation of the angular dependence of the emitted harmonic yield $I^{n \textrm{th}}_{\mathrm{total}}$ without Terahertz (THz) fields, where we analyze the intrinsic symmetry of graphene, Figures~\ref{fig:SI_polar_mir} illustrate the computed angular dependence of the emitted harmonic yield $I^{n \textrm{th}}_{\mathrm{total}}$ using only the Mid-Infrared (MIR) field. The emitted harmonics exhibit a six-fold symmetry, mirroring the hexagonal lattice symmetry of graphene. As depicted in Fig.~\ref{fig:SI_polar_mir}, the lower-order harmonics display an almost circular angular dependence, indicative of the circular symmetry inherent in Dirac cones. In contrast, the higher-order harmonics demonstrate a more intricate six-fold symmetry in their angular dependence, owing to the deviation of the electronic structure of graphene from a simple Dirac cone when a single-particle energy is distant from the Dirac point.
\begin{figure}[tb]
\centering
\includegraphics[width=0.50\linewidth]{pic/polar_mir.pdf}
\caption{\label{fig:SI_polar_mir}
The angular dependence of the harmonic yield obtained from the electron dynamics calculations in the presence of the MIR field. The third, fifth, and seventh harmonic yields are scaled by factors of 60, 800, and 1000, respectively.
}
\end{figure}
Similarly, we adopt the subsequent expression for the Terahertz (THz) pulse:
\begin{align}
\vecb A_{THz}(t) = -\frac{E_{THz}}{\omega_{THz}} \vecb{e}{THz}
\sin(\omega{THz} t) \cos^4 \left (\frac{\pi}{T_{THz}} t \right)
\label{eqn:laser_pulse}
\end{align}
within the interval $-T_{THz}/2 < t < T_{THz}/2$, and zero outside this range. Here, $E_{THz}$ denotes the peak strength of the THz field, $\omega_{THz}$ is the mean frequency, $\vecb e_{THz}$ represents a unit vector along the polarization direction, and $T_{THz}$ stands for the pulse duration. In our investigation, the pulse duration $T_{THz}$ is fixed at 40 ps, and the mean frequency $\omega_{THz}$ is set to 1.2407 meV/$\hbar$. The time profile of the applied THz electric field is depicted in the inset of Fig.\ref{fig:current}(a).

To illuminate the intricacies of Terahertz (THz)-assisted Mid-Infrared (MIR)-induced High Harmonic Generation (HHG) in graphene, we conduct an electron dynamics calculation in the presence of both THz and MIR fields, denoted as $\vecb E_{THz}(t) + \vecb E_{MIR}(t)$. Here, we set $E_{MIR}$ to 6.5 MV/cm and $E_{THz}$ to 0.5 MV/cm. It is pertinent to note that experimentally available intense THz pulses can exhibit amplitudes exceeding 1 MV/cm~\cite{10.1063/1.3560062}. The polarization direction of the THz field $\vecb e_{THz}$ is aligned with the $\Gamma$--$M$ direction (the $x$-direction in our setup), while the polarization direction of the MIR field $\vecb e_{MIR}$ is considered as a tunable parameter. Figures~\ref{fig:current}(a) and (b) depict the computed current $\vecb J(t)$ induced by $\vecb E_{THz}(t) + \vecb E_{MIR}(t)$ as a function of time. The result for the parallel configuration ($\vecb e_{MIR} = \vecb e_x = \vecb e_{THz}$) is presented in Fig.\ref{fig:current}(a), while the result for the perpendicular configuration ($\vecb e_{MIR} = \vecb e_y \perp \vecb e_{THz}$) is shown in Fig.\ref{fig:current}(b). The $x$ and $y$ components are represented by blue and red lines, respectively. Evidently, Figs.\ref{fig:current}~(a) and (b) illustrate that the THz field induces a current on the picosecond time scale, whereas the MIR field induces a current on a much shorter time scale.
\begin{figure*}[ht]
\includegraphics[width=0.9\linewidth]{pic/fig1.pdf}
\caption{\label{fig:current}
(a, b)~The current $\vecb{J} (t)$ induced by THz and MIR fields, $\vecb E_{THz}(t)+\vecb E_{MIR}(t)$. The $x$-component of the current is shown as the blue line, whereas the $y$-component is shown as the red line. The inset is the panel (a) shows the time profile of the applied THz field. (c, d)~The current $\vecb{J}(t)$ induced by the static and MIR fields, $\vecb E_{dc}(t)+\vecb E_{MIR}(t-\tau_{MIR})$. The $x$ component of the current is shown as the orange line, whereas the $y$-component is shown as the green line. In the panels~(a) and (c), the polarization of all the fields is parallel to the $\Gamma$--$M$ direction (the $x$-direction in the present setup) as $\vecb e_{THz}=\vecb e_{dc}=\vecb e_{MIR}=\vecb e_x$. In the panels~(b) and (d), the polarization of THz and static fields is parallel to the $x$-direction as $\vecb e_{THz}=\vecb e_{dc}=\vecb e_{MIR}=\vecb e_x$, while that of the MIR field is perpendicular as $\vecb e_{MIR}=\vecb e_y$. (e) The power spectra $I_{\mathrm{HHG}}(\omega)$ computed using the current in (a) and (c). (f) The power spectra $I_{\mathrm{HHG}}(\omega)$ computed using the current in (b) and (d).
}
\end{figure*}

To delve into Mid-Infrared (MIR)-induced High Harmonic Generation (HHG) amidst the presence of Terahertz (THz) fields, we scrutinize the current induced by the MIR field under the influence of the THz field. For this analysis, we delineate two types of currents. Firstly, we denote the current induced by both the THz and MIR fields as $\vecb J^{THz + MIR}(t)$. Secondly, we denote the current induced solely by the THz field as $\vecb J^{THz}(t)$. Defining the current induced by the MIR field in the presence of the THz field as $\vecb J^{eff}(t) = \vecb J^{THz + MIR}(t) - \vecb J^{THz}(t)$, we subject it to Fourier transformation, followed by computation of the power spectrum of the emitted harmonics using Eq.(\ref{eqn:spectrum}). The solid line in Fig.\ref{fig:current}(e) showcases the power spectrum computed utilizing the current $\vecb J(t)$ depicted in Fig.\ref{fig:current}(a), wherein the polarization directions of the THz and MIR fields are parallel. Conversely, the solid line in Fig.\ref{fig:current}(f) exhibits the power spectrum computed employing the current $\vecb J(t)$ illustrated in Fig.\ref{fig:current}(b), where the polarization directions of the two fields are perpendicular. Notably, Fig.\ref{fig:current}(e) elucidates the generation of second and higher even-order harmonics alongside odd-order harmonics, attributed to the local breakdown of the system's inversion symmetry induced by the THz field. This phenomenon, known as electric-field-induced second-harmonic generation (EFISH) or THz-induced second-harmonic generation (TFISH), has been extensively documented\cite{PhysRevLett.8.404,PhysRev.137.A801,Nahata:98,COOK1999221}. Similarly, even-order harmonics are generated in the perpendicular configuration ($\vecb e_{MIR} \perp \vecb e_{THz}$), as depicted in Fig.\ref{fig:current}(f).

Incorporating the THz pulse explicitly in the electron dynamics computation extends the propagation time (42 ps in the current scenario), as illustrated in Figs.\ref{fig:current}(a) and (b). Consequently, performing electron dynamics calculations with the explicit inclusion of THz pulses entails a substantial computational burden. To alleviate the computational overhead associated with modeling Mid-Infrared (MIR)-induced High Harmonic Generation (HHG) in graphene under a THz field, we adopt a static field approximation based on the quasistatic approximation~\cite{PhysRevB.106.024313}.

For practical analysis, we conduct two simulations. In the first simulation, electron dynamics are computed under a static field $\vecb E_{dc}(t) = \vecb e_{dc} E_{dc}\Theta(t)$, abruptly initiated at $t=0$. Here, $\vecb e_{dc}$ represents the unit vector along the polarization direction of the static field, and $E_{dc}$ denotes the field strength. Upon the abrupt activation of the static field, the induced electron dynamics prompt a current. Following a sufficiently prolonged time of propagation, the driven system attains a steady state, and the current stabilizes over time. We designate the current induced solely by $\vecb E_{dc}(t)$ as $\vecb J^{dc}(t)$.

In the second simulation, electron dynamics are computed under both the MIR and static fields, $\vecb E_{dc}(t) + \vecb E_{MIR}(t - \tau_{MIR})$, where the pulse center of the MIR field is shifted by $\tau_{MIR}$. We denote the current induced by $\vecb E_{dc}(t) + \vecb E_{MIR}(t - \tau_{MIR})$ as $\vecb J^{dc+MIR}(t)$. The shift $\tau_{MIR}$ can be made sufficiently large to investigate the MIR-induced electron dynamics for a full nonequilibrium steady state realized by the static field $\vecb E_{dc}(t)$. Subsequently, the MIR-induced current can be extracted as $\vecb J^{eff}(t) = \vecb J^{dc+MIR}(t) - \vecb J^{dc}(t)$ to analyze MIR-induced HHG in the presence of the static field.

Figures~\ref{fig:current}(c) and (d) depict the current $\vecb J^{dc+MIR}(t)$ induced by both the static and MIR fields. The orange and green lines represent the $x$ and $y$ components of the current, respectively. Here, the static field aligns with the $\Gamma$--$M$ direction (the $x$-direction in our setup), and its strength $E_{dc}$ matches the peak strength of the THz field, $E_{dc}=E_{THz}=0.5$MV/cm. In Fig.\ref{fig:current}(c), the MIR field aligns parallel to the static field, while in Fig.\ref{fig:current}(d), it aligns perpendicular to the static field. To incorporate the MIR field into the nonequilibrium steady-state under the static field, we set the time delay $\tau_{MIR}$ of the MIR field to 1~ps, exceeding the relaxation time scales of the quantum master equation, $T_1$ and $T_2$.

To investigate HHG in the presence of the static field $\vecb E_{dc}(t)$, we extract the current
$\vecb J^{eff}(t)$ induced by the MIR field in the presence of the static field by subtracting
$\vecb J^{dc}(t)$ from $\vecb J^{dc+MIR}(t)$: $\vecb J^{eff}(t)=\vecb J^{dc+MIR}(t)-\vecb
J^{dc}(t)$. The dashed lines in Figs~\ref{fig:current}(e) and (f) represent the HHG spectra
computed using the current shown in Figs\ref{fig:current}(c) and (d), respectively. Remarkably, the
results obtained through the quasistatic approximation with a static field align perfectly with
those computed by explicitly including the THz pulse. This consistency underscores the validity of
the quasistatic approximation for analyzing HHG under MIR and THz fields. Additionally, we verified
the consistency of the quasistatic approximation across various static field strengths (refer to
Figure~\ref{fig:SI_polar_mir}). Henceforth, we employ the static field within the quasistatic approximation rather than explicitly incorporating the THz pulse. 

\begin{figure}[tb]
\includegraphics[width=0.9\linewidth]{pic/SI_t1t2.pdf}
\caption{\label{fig:intensity_relaxation}
The harmonic yields are shown as a function of the static field strength $E_{dc}$. In each panel, the results obtained using the different relaxation times, $T_1$ and $T_2$, are compared. The results of the third harmonics are shown in the panels~(a) and (b), whereas those of the fifth harmonics are shown in the panels~(c) and (d). The results obtained using the parallel configuration ($\vecb e_{MIR}=\vecb e_x = \vecb e_{THz}$) are shown in the panels~(a) and (c), whereas those obtained using the perpendicular configuration ($\vecb e_{MIR}=\vecb e_y \perp \vecb e_{THz}$) are shown in the panels~(b) and (d).
}
\end{figure}

We further investigate the influence of relaxation times, $T_1$ and $T_2$, on HHG in the presence
of THz and MIR fields. Employing the methodologies outlined in Section.~\ref{sec:master}, we compute the yields of third- and fifth-order harmonics under varying relaxation times. The results, depicted in Fig.\ref{fig:intensity_relaxation}, exhibit consistent qualitative trends in HHG enhancement with THz field irradiation across different relaxation times. Thus, the specific choice of relaxation times does not substantially alter the enhancement phenomenon.

Relaxation times are determined by diverse scattering mechanisms, including electron-electron,
electron-phonon, and electron-impurity interactions. Consequently, the actual relaxation times in
practical settings depend on experimental conditions. Nonetheless, the findings presented in
Fig.~\ref{fig:intensity_relaxation} suggest that HHG enhancement via THz field irradiation can
manifest as a robust phenomenon across a broad spectrum of experimental scenarios.

\begin{figure}[tb]
\includegraphics[width=0.9\linewidth]{pic/SI_hhg.pdf}
\caption{\label{fig:quasi-static}
The power spectra of emitted harmonics, $I_{HHG}(\omega)$, are shown. The results obtained using a weak THz field ($E_{THz}=0.1$~MV/cm) are shown in the panels~(a) and (b), while those obtained using a strong THz field ($E_{THz}=1.0$~MV/cm) are shown in the panels~(c) and (d). The results obtained using the parallel configuration ($\vecb e_{MIR}=\vecb e_x = \vecb e_{THz}$) are shown in the panels~(a) and (c), whereas those obtained using the perpendicular configuration ($\vecb e_{MIR}=\vecb e_y \perp \vecb e_{THz}$) are shown in the panels~(b) and (d).
}
\end{figure}

 we extend our inquiry to validate the quasi-static approximation across varying strengths of the THz field. We repeat the analyses presented in Figs.\ref{fig:current}(e) and (f) while altering the THz field strength. Results obtained under a weak THz field ($E_{THz}=0.1$MV/cm) are depicted in Figs.\ref{fig:quasi-static}(a) and (b), while those under a strong THz field ($E_{THz}=1.0$MV/cm) are shown in Figs.\ref{fig:quasi-static}(c) and (d). As observed from the figures, the outcomes of the quasi-static approximation closely mirror those obtained from calculations with THz laser pulses across all investigated field strengths and polarization directions. Hence, we affirm the efficacy of the quasi-static approximation in accurately describing electron dynamics in graphene under THz and MIR fields, encompassing both weak and strong field regimes.
This agreement between the quasistatic approximation and the explicit inclusion of the THz pulse underscores the pivotal role played by the nonequilibrium steady state under the static field in MIR-induced HHG in graphene in the presence of a THz field.
%=======================================================================================================================================================================
\section{Orientational Dependence of HHG}
%=======================================================================================================================================================================
\begin{figure*}[ht]
\centering
\includegraphics[width=1.0\linewidth]{pic/polar.pdf}
\caption{\label{fig:polar}
The angular dependence of the harmonic yield in the nonequilibrium steady-states under a static field along the $\Gamma$--$M$ direction is shown for different static field strengths, $E_{dc}$. The angle $\theta$ denotes the relative angle between the static field and the $MIR$ field. (a--d) The total intensity $I^{n \textrm{th}}_{\mathrm{total}}$ is shown for the second, third, fourth, and fifth harmonics. (e-h) The component of the intensity parallel to $\vecb e_{MIR}$ is shown for each harmonic. (i-l) The component of the intensity perpendicular to $\vecb e_{MIR}$ is shown for each harmonic. The results are normalized by the maximum total intensity $I^{n \textrm{th}}_{\mathrm{total}}$ for each harmonic.
}
\end{figure*}

We explore high-harmonic generation (HHG) in graphene within the quasi-static approximation, varying the relative angle between the static and mid-infrared (MIR) fields. For our analysis, we maintain the direction of the static field $\vecb e_{dc}$ along the $\Gamma$--$M$ axis (the $x$-axis in our setup) and set the peak field strength of the MIR field $E_{MIR}$ to 6.5~MV/cm. We investigate the emitted harmonics by manipulating the polarization direction of the MIR field, $\vecb e_{MIR}$, and adjusting the strength of the static field, $E_{dc}$.

Figures~\ref{fig:polar}(a--d) illustrate the angular dependence of the emitted harmonic yield $I^{n \textrm{th}}$ for various harmonic orders. Here, $\theta$ represents the relative angle between the MIR and static fields. In Fig.\ref{fig:polar}(a), absence of a static field results in no second harmonic generation, given the intrinsic inversion symmetry of graphene. However, with the introduction of a static field, second harmonics emerge due to the breakdown of this symmetry. Notably, for a static field strength of 0.5MV/cm, the emitted second-harmonic intensity peaks at approximately 45$^{\circ}$ relative angle.

In Fig.\ref{fig:polar}(b), the third-harmonic yield appears nearly isotropic (depicted by the black
line) in the absence of a static field, reflecting the rotational symmetry of the Dirac cone (refer
also to Figure~\ref{fig:SI_polar_mir}). Conversely, under the influence of a strong static field ($E_{dc}=1.0$MV/cm), the third-harmonic intensity exhibits significant angular dependence: it is notably enhanced when the static and MIR fields are perpendicular to each other, whereas it is suppressed for parallel field orientations. This enhancement for the perpendicular configuration can be attributed to the coupling between the intraband transition induced by the static field and the interband transition induced by the MIR field, as previously suggested\cite{PhysRevB.103.L041408}.

The angular dependence of higher-order harmonics becomes more intricate under a static field, as
depicted in Figs.\ref{fig:polar}(c) and (d). Notably, the fifth-order harmonic emission shows
significant enhancement in the presence of either static or THz fields (Fig.\ref{fig:polar}(d)).
For instance, applying a static field of 0.5MV/cm boosts the fifth-order harmonic intensity by more
than tenfold compared to that induced solely by the MIR field (indicated by the green line in
Fig.\ref{fig:polar}(d)). This enhancement ratio surpasses that observed for the third-order
harmonic, suggesting a greater potential for field-induced enhancement in higher-order harmonics.
Indeed, the seventh-order harmonic exhibits a 25-fold enhancement with a static field strength of
0.5MV/cm (refer to Figure~\ref{fig:SI_polar_mir}).

To further elucidate the angular dependence of HHG in graphene, we decompose the harmonic intensity $I_{\mathrm{HHG}}(\omega)$ into parallel and perpendicular components with respect to the polarization of the driving MIR field. The parallel component of the HHG intensity is defined as 
\begin{align}
I^{\textrm{para}}_{\mathrm{HHG}}(\omega)\sim \omega^2 \left | \int^{\infty}_{-\infty} dt \vecb e_{MIR} \cdot \vecb J(t) e^{i\omega t} \right |^2,
\label{eqn:spectrum-para}
\end{align}
where $\vecb e_{MIR}$ is the unit vector along the polarization direction of the MIR field. Likewise, the perpendicular component is defined as
\begin{align}
I^{\textrm{perp}}_{\mathrm{HHG}}(\omega)\sim \omega^2 \left | \int^{\infty}_{-\infty} dt \bar{\vecb e}_{MIR} \cdot \vecb J(t) e^{i\omega t} \right |^2,
\label{eqn:spectrum-perp}
\end{align}
where $\bar{\vecb e}_{MIR}$ is a unit vector perpendicular to $\vecb e_{MIR}$, i.e., $\bar{\vecb e}_{MIR} \cdot\vecb e_{MIR}=0$. The total intensity $I_{\textrm{HHG}}$ in Eq.~(\ref{eqn:spectrum}) is reproduced by the sum of $I^{\textrm{para}}_{\mathrm{HHG}}(\omega)$ and $I^{\textrm{perp}}_{\mathrm{HHG}}(\omega)$ as $I_{\mathrm{HHG}}(\omega)=I^{\textrm{para}}_{\mathrm{HHG}}(\omega)+I^{\textrm{perp}}_{\mathrm{HHG}}(\omega)$.

Equations (\ref{eqn:spectrum-para}) and (\ref{eqn:spectrum-perp}) are utilized to dissect the emitted harmonic intensity into parallel and perpendicular components. Figures~\ref{fig:polar}~(e--h) and (i--l) elucidate the angular dependence of the parallel and perpendicular components of the harmonic intensity, respectively, for different orders.

In Figs.\ref{fig:polar}(a), (e), and (i), the parallel component of the second harmonic under the static field peaks around 45$^{\circ}$, constituting the dominant contribution to the total second-harmonic intensity at this orientation. Conversely, the maximum perpendicular component is consistently achieved when the MIR and static fields are orthogonal to each other. In Figs.\ref{fig:polar}(b), (f), and (j), the third harmonic is predominantly governed by its parallel component across all angles and static field strengths examined. Notably, for both second- and third-harmonic generation, the parallel components prevail when the emitted harmonic intensity is maximized.

Qualitative distinctions emerge between the lower-order harmonics (second and third) and the
higher-order ones (fourth and fifth). In Fig.\ref{fig:polar}(c), the fourth harmonic yield peaks at
an angle $\theta$ of 90$^{\circ}$ under the strongest applied static field, $E_{dc}=1.0$MV/cm. A
comparison of Figs.\ref{fig:polar}(g) and (k) reveals the predominance of the perpendicular
component in the emitted harmonic intensity under these conditions. Conversely, as depicted in
Figs.\ref{fig:polar}(d), (h), and (l), the emitted fifth harmonic at the most efficient angle is
primarily governed by the perpendicular component, despite the dominance of the parallel component
at all angles in the absence of a static field. Hence, the emission pathways associated with the
perpendicular components are anticipated to play a crucial role in enhancing MIR-induced HHG by a
THz field. This trend persists for higher-order harmonics as well (see Figure\ref{fig:SI_polar_mir}).
%=======================================================================================================================================================================
\section{Comparison of Nonequilibrium Steady State and Thermodynamic Model}
%=======================================================================================================================================================================
\begin{figure}[ht]
\includegraphics[width=0.9\linewidth]{pic/3rd_hh_vs_thermal.pdf}
\includegraphics[width=0.9\linewidth]{pic/5th_hh_vs_thermal.pdf}
\includegraphics[width=0.9\linewidth]{pic/7th_hh_vs_thermal.pdf}
\caption{\label{fig:intensity_tem}
The emitted light intensity, $I^{\textit{n}th}$, is shown as a function of the excess energy for (a) third (b) fifth, and (c) seventh harmonics. The results for the nonequilibrium steady-states induced by a static field parallel (red solid line) and perpendicular (blue dashed line) to the MIR field are compared with the thermodynamic model (green dotted line). In each panel, the field strength of the static field parallel to the MIR field is shown as the secondary axis.
}
\end{figure}

In this investigation, we delve into the role of nonequilibrium steady states in high harmonic generation (HHG) by juxtaposing the outcomes of the quasistatic approximation with those derived from the thermodynamic model~\cite{mics2015thermodynamic}, a framework previously utilized in studying HHG in graphene under THz fields~\cite{Hafez2018,doi:10.1126/sciadv.abf9809}. The quasistatic approximation replaces the THz pulse with a static field to describe the electronic system's behavior under THz irradiation, whereas the thermodynamic model approximates the system's response to a THz pulse as a high-temperature thermal state~\cite{mics2015thermodynamic}. The discrepancy between these models elucidates the influence of nonequilibrium distributions, delineating their significance in HHG.

The quasistatic approximation is characterized by the static field strength, $E_{dc}$, while the thermodynamic model relies on the electron temperature $T_e$. To enable a direct comparison between these models, we introduce the concept of excess energy~\cite{PhysRevB.106.024313} as a common metric of excitation intensity. The excess energy under the quasistatic approximation, denoted as $\Delta E^{\mathrm{non-eq}}{\textrm{excess}}(E{dc})$, quantifies the change in total energy due to the static field $\vecb E_{dc}(t)$. Conversely, the excess energy within the thermodynamic model, $\Delta E^{\textrm{thermo}}_{\textrm{excess}}(T_e)$, measures the energy change arising from an increase in electron temperature from room temperature ($T_e=300$ K). This common excess energy facilitates an objective and quantitative comparison between the two models~\cite{PhysRevB.106.024313}.

Figure~\ref{fig:intensity_tem} illustrates the comparison between the results obtained from the quasistatic approximation and the thermodynamic model. Setting the MIR field strength to 6.5 MV/cm and its polarization direction to the $\Gamma$--$M$ direction (the $x$-axis), we observe distinct behaviors in odd-order harmonics between the two models. Figure~\ref{fig:intensity_tem}(a) showcases the substantial enhancement and suppression of the MIR-induced third harmonic under the quasistatic approximation for parallel and perpendicular configurations, respectively. In contrast, the thermodynamic model yields nearly constant results. Figures\ref{fig:intensity_tem}~(b) and (c) further reveal significant enhancements in the fifth- and seventh-harmonic yields under a static field within the quasistatic approximation, while the thermodynamic model exhibits marginal variations in harmonic yields with increasing electron temperature. Consequently, the observed HHG enhancement cannot be solely attributed to the simple heating of electronic systems within the thermodynamic model, underscoring the crucial role of non-equilibrium electronic dynamics induced by the field. The minimal alterations in harmonic yields within the thermodynamic model relative to those predicted by the nonequilibrium steady-state model suggest that modifications in the population distribution around the Fermi level exert negligible influence on HHG spectra.
%=======================================================================================================================================================================
\section{Contribution of Nonequilibrium Population}
%=======================================================================================================================================================================
\begin{figure}[ht]
\includegraphics[width=0.9\linewidth]{pic/pop_dist.pdf}
\includegraphics[width=0.9\linewidth]{pic/pop_c.pdf}
\caption{\label{fig:pop}
(a) The calculated conduction population distribution, $n^{\mathrm{neq-steady}}_{c\vecb k}$ for the nonequilibrium steady-state is shown. Here, the Dirac point is indicated by the blue circle. (b--e) The angular dependence of the emitted harmonic intensity is shown for the (b) second, (c) third, (d) fourth, and (e) fifth harmonics. The results obtained using the nonequilibrium population model and the nonequilibrium steady-state are shown by the blue and green solid lines, respectively.
}
\end{figure}

In our exploration of the coherent coupling between the MIR and THz fields, beyond the population contribution induced by the THz field, we introduce a nonequilibrium population distribution model as an extension of the thermodynamic model.

Within the thermodynamic model, the THz field's contribution is represented by adjusting the population distribution via an increased electronic temperature in the reference Fermi--Dirac distribution. This model captures only the population contribution, corresponding to the diagonal elements of the density matrix, based on the thermal distribution.

To delve into the coherent coupling contribution, we extend the thermodynamic model by substituting the reference Fermi--Dirac distribution in the relaxation operator (Eq. (\ref{eqn:relaxation})) with the population distribution of the nonequilibrium steady state under a static field. This extension incorporates the population contribution, signified by diagonal elements of the density matrix, while omitting THz-induced coherence, represented by the off-diagonal elements of the density matrix.

By comparing the nonequilibrium population model with the fully dynamical model, which encompasses both population and coherence effects, we can discern the role of coherent coupling between the THz and MIR fields. This comparative analysis helps elucidate the distinct contributions of population and coherence effects to HHG, providing valuable insights into the underlying mechanisms governing this phenomenon.
To formulate the nonequilibrium population model, we first analyze the population distribution in the nonequilibrium steady state under a static field. The population distribution in the Brillouin zone can be expressed as
\begin{align}
n_{b \vecb k}(t) & = \int d \vecb k' \delta(\vecb k - \vecb K'(t)) \mathrm{Tr}\left [
| u^H_{b\vecb k'}(t)\rangle \langle u^H_{b\vecb k'}(t)| \rho_{\vecb k'}(t) 
\right ] \nonumber \\
&=\langle u^H_{b,\vecb k-e\vecb A(t)}(t)| \rho_{\vecb k-e\vecb A(t)}(t) 
| u^H_{b,\vecb k-e\vecb A(t)}(t)\rangle,
\end{align}
where $\vecb K'(t)$ is the accelerated wavevector in accordance with the acceleration theorem, $\vecb K'(t)=\vecb k'+e\vecb A(t)$. The population distribution in the nonequilibrium steady state can be evaluated in the long-time propagation limit under a static field $\vecb A(t)= -\vecb E_{dc}t$,
\begin{align}
n^{\mathrm{neq-steady}}_{b\vecb k} = \lim _{t\rightarrow \infty } n_{b \vecb k}(t).
\end{align}

In Fig.\ref{fig:pop}(a), we illustrate the population distribution in the conduction band for the nonequilibrium steady-state under a static field with a strength of $E_{dc}=0.5$~MV/cm. The static field is oriented along the $\Gamma$--$M$ direction ($x$-axis), and the blue circle marks the Dirac point ($K$ point).

In this depiction, the region to the left of the Dirac point is predominantly occupied by the field-induced population in the nonequilibrium steady-state, while the region to the right of the Dirac point appears nearly empty. This asymmetry disrupts the inversion symmetry of the system. We utilize this nonequilibrium population distribution as the reference distribution of the relaxation operator in Eq.~(\ref{eqn:relaxation}) instead of the Fermi--Dirac distribution to establish the nonequilibrium population model.

In Fig.\ref{fig:pop}(b), we present the angular dependence of the second-harmonic yield under a static field with a strength of $E_{dc}=0.5$MV/cm. The corresponding angular dependences of the third, fourth, and fifth harmonics are depicted in Figs.\ref{fig:pop}(c--e), respectively. Each panel displays results obtained using the nonequilibrium population model as the blue solid line, juxtaposed with results derived from the quasi-static approximation, depicted as the green solid line, which matches the result shown in Fig.\ref{fig:polar}.

Figs~\ref{fig:pop}~(b) and (d) highlight that even-order harmonics computed with the nonequilibrium population model are notably weaker compared to those calculated using the fully dynamical approach based on the quasi-static approximation. This discrepancy indicates that under the charge-neutral condition ($\mu=0$) examined here, the resonant effects of the MIR field at two- and four-photon resonances are significantly distant from the Fermi level. Consequently, modifications to the population near the Fermi surface yield minor contributions to even-order harmonic generation. In contrast, the fully dynamical calculation reveals that the THz field can coherently couple with the MIR field via off-diagonal elements of the density matrix, enabling coherent coupling not only around the Fermi level but also across the Brillouin zone wherever dipole transitions are permitted. Thus, the coherent coupling component may strengthen even-order harmonic generation by enhancing contributions from resonant quantum pathways.

Fig.\ref{fig:pop}(c) demonstrates that the third-harmonic yield computed using the fully dynamical model is $1.57$ times stronger than that obtained using the nonequilibrium population model when the fields are perpendicular. This result suggests that the THz field enhances third-harmonic generation for the perpendicular configuration, with both coherent coupling and incoherent population playing crucial roles in this THz-assisted enhancement mechanism. Conversely, when the fields are parallel, the third-harmonic yield calculated using the fully dynamical approach is $0.57$ times weaker than that computed using the nonequilibrium population model. This finding indicates that contributions from coherent coupling and incoherent population counteract each other, diminishing the overall signal. Thus, both coherent coupling and incoherent population influence third-harmonic generation depending on the relative angle $\theta$ between the THz and MIR fields.

In Fig.\ref{fig:pop}(e), we observe that the fifth-order harmonic yield computed using the fully dynamical model is significantly higher than that obtained using the nonequilibrium population model, except when the MIR and THz fields are parallel. This observation suggests that coherent coupling predominantly contributes to the enhancement of fifth-harmonic generation for most angles, although both coherent coupling and incoherent population effects are relevant when the fields are parallel. These consistent results are similarly observed for higher-order harmonics (see Appendix~\ref{sec:angle-dep-higher}).

In summary, when only a MIR field is applied to graphene, the induced HHG is attributed to interference between multiple excitation pathways involving nonlinear coupling between MIR-induced intraband and interband transitions. Conversely, the substantial enhancement of HHG observed in the presence of THz fields indicates the activation of an additional nonlinear coupling mechanism. This mechanism arises from coherent coupling between MIR- and THz-induced transitions, suggesting its predominance over other processes in contributing to overall harmonic yield.

The comparison between the results obtained using the fully dynamical calculation and the nonequilibrium population distribution model has provided valuable insights into the roles of coherent coupling between the MIR and THz fields. The dominance of coherent coupling in generating THz-induced even-order harmonics and enhancing high-order harmonics suggests its crucial role in driving nonlinear optical processes in solids. Conversely, the enhancement of third harmonics under a THz field is influenced by both coherent coupling and the nonequilibrium population. Furthermore, coherent coupling appears to predominantly contribute to the enhancement of higher-order harmonics.

Importantly, these enhancement mechanisms are not confined to specific laser parameters but can be realized under more general conditions. Therefore, effective control of both coherent coupling and population dynamics becomes essential for boosting HHG from solids using multicolor laser fields.

Furthermore, the significance of coherent coupling extends across various orders of harmonic generation, as evidenced by the coherent coupling mechanism's influence on low-order harmonic phenomena (see Fig.~\ref{fig:pop}). This underscores the indispensability of field-induced coherence in nonlinear optical effects more broadly. Consequently, these findings hint at the potential for efficiently controlling electron and spin dynamics through coherent coupling, leveraging multi-color lasers. Such capabilities would transcend mere frequency conversion of light, opening avenues for the advancement of ultrafast optoelectronics and optospintronics.
